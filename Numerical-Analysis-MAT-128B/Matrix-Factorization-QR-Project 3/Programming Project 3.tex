\documentclass[final,12pt,reqno]{amsart}

\setlength{\textwidth }{7.50 in}
\setlength{\textheight}{9.25 in}
\setlength{\oddsidemargin }{0.00 in}
\setlength{\evensidemargin}{0.00 in}
\setlength{\oddsidemargin }{0.00 in}
\setlength{\evensidemargin}{0.00 in}
\setlength{\hoffset}{-0.50 in}
\setlength{\voffset}{-0.50 in}
\setlength{\headsep}{12 pt}
\setlength{\headheight}{40.53336 pt}
\setlength{\topmargin }{00 pt}
\setlength{\footskip}{0.50 in}
\setlength{\parskip}{12 pt}
\setlength{\parindent}{00 pt}
\setlength{\fboxsep}{10 pt}

\usepackage{graphicx}
\DeclareGraphicsExtensions{.png}
\graphicspath{{C:/Users/Christopher/Desktop/"MAT 128B"/"Project 3"}}

\usepackage{enumerate}
\usepackage{multicol}
\usepackage{bm}
\usepackage{color}

\usepackage[bookmarks=true]{hyperref}
\usepackage{bookmark}

\usepackage{verbatim}
\usepackage{slashbox}

\font\myfiverm=cmr5 scaled 500

\usepackage{fancyhdr}
\pagestyle{fancy}
\fancyhead{}
\fancyhead[LO]{}
\fancyhead[LE]{}
\fancyhead[CO]{\textbf{Programming Project 3}}
\fancyhead[CE]{\textbf{Programming Project 3}}
\fancyhead[RO]{\textbf{Christopher Wong\\
                        999234204\\
												MAT 128B\\
                        3/8/2017}}
\fancyhead[RE]{\textbf{Christopher Wong\\
                        999234204\\
												MAT 128B\\
                        3/8/2017}}
												
\newcommand\abs[1]{\left|#1\right|}
\renewcommand{\arraystretch}{1.2}

%%%%%%%%%%%%%%%%%%%%%%% START OF DOCUMENT %%%%%%%%%%%%%%%%%%%%%%%

\begin{document}

\thispagestyle{fancy}

\pdfbookmark[2]{(a)}{parta}
\textbf{(a)}

\[
A =
\left[\begin{array}{@{}*{4}{c}@{}}
  3 & -1 & -1 & 0\\
	2 & 0 & 2 & 2\\
	-1 & 1 & 1 & 1\\
	0 & 0 & -2 & -1\\
	3 & -1 & 1 & 1\\
	1 & -1 & -3 & -2
  \end{array} \right]
\]

\textbf{\underline{TOL = 10^{-16}}}

\[
Q =
\left[\begin{array}{@{}*{4}{c}@{}}
    -5.0e-01&     3.535533905932737e-01&    -3.535533905932737e-01&     1.543033499620919e-01\\
                         0&     7.071067811865475e-01&     1.570092458683775e-16&     6.172133998483677e-01\\
     5.0e-01&     3.535533905932737e-01&    -3.535533905932737e-01&     1.543033499620919e-01\\
                         0&                         0&    -7.071067811865475e-01&    -6.172133998483675e-01\\
    -5.0e-01&     3.535533905932737e-01&     3.535533905932738e-01&     3.086066999241838e-01\\
    -5.0e-01&    -3.535533905932737e-01&    -3.535533905932738e-01&    -3.086066999241838e-01
  \end{array} \right]
\]
\[
R_1 =
\left[\begin{array}{@{}*{4}{c}@{}}
     2.0e+00&    -4.0e+00&     2.0e+00&     1.0e+00\\
                         0&     2.828427124746190e+00&     2.828427124746190e+00&     2.828427124746190e+00\\
                         0&                         0&     2.828427124746190e+00&     1.414213562373095e+00\\
                         0&                         0&                         0&     3.597533769998863e-16
  \end{array} \right]
\]
\[
R_2 = \text{Empty matrix: 4-by-0}
\]
\[
P = 
\left[\begin{array}{@{}*{4}{c}@{}}
     0&     1&     0&     0\\
     1&     0&     0&     0\\
     0&     0&     1&     0\\
     0&     0&     0&     1
  \end{array} \right]
\]

$\|Q^TQ-I\|_2$ = 9.385906354489062e-01

\textbf{\underline{TOL = 10^{-14}}}

\[
Q =
\left[\begin{array}{@{}*{3}{c}@{}}
    -5.000000000000000e-01&     3.535533905932737e-01&    -3.535533905932737e-01\\
                         0&     7.071067811865475e-01&     1.570092458683775e-16\\
     5.000000000000000e-01&     3.535533905932737e-01&    -3.535533905932737e-01\\
                         0&                         0&    -7.071067811865475e-01\\
    -5.000000000000000e-01&     3.535533905932737e-01&     3.535533905932738e-01\\
    -5.000000000000000e-01&    -3.535533905932737e-01&    -3.535533905932738e-01
  \end{array} \right]
\]
\[
R_1 =
\left[\begin{array}{@{}*{3}{c}@{}}
     2.000000000000000e+00&    -4.000000000000000e+00&     2.000000000000000e+00\\
                         0&     2.828427124746190e+00&     2.828427124746190e+00\\
                         0&                         0&     2.828427124746190e+00
  \end{array} \right]
\]
\[
R_2 =
\left[\begin{array}{@{}*{1}{c}@{}}
     1.000000000000000e+00\\
     2.828427124746190e+00\\
     1.414213562373095e+00
  \end{array} \right]
\]
\[
P = 
\left[\begin{array}{@{}*{4}{c}@{}}
     0&     1&     0&     0\\
     1&     0&     0&     0\\
     0&     0&     1&     0\\
     0&     0&     0&     1
  \end{array} \right]
\]

$\|Q^TQ-I\|_2$ = 4.224272170198187e-16

\textbf{\underline{TOL = 10^{-12}}}

\[
Q =
\left[\begin{array}{@{}*{3}{c}@{}}
    -5.000000000000000e-01&     3.535533905932737e-01&    -3.535533905932737e-01\\
                         0&     7.071067811865475e-01&     1.570092458683775e-16\\
     5.000000000000000e-01&     3.535533905932737e-01&    -3.535533905932737e-01\\
                         0&                         0&    -7.071067811865475e-01\\
    -5.000000000000000e-01&     3.535533905932737e-01&     3.535533905932738e-01\\
    -5.000000000000000e-01&    -3.535533905932737e-01&    -3.535533905932738e-01
  \end{array} \right]
\]
\[
R_1 =
\left[\begin{array}{@{}*{3}{c}@{}}
     2.000000000000000e+00&    -4.000000000000000e+00&     2.000000000000000e+00\\
                         0&     2.828427124746190e+00&     2.828427124746190e+00\\
                         0&                         0&     2.828427124746190e+00
  \end{array} \right]
\]
\[
R_2 =
\left[\begin{array}{@{}*{1}{c}@{}}
     1.000000000000000e+00\\
     2.828427124746190e+00\\
     1.414213562373095e+00
  \end{array} \right]
\]
\[
P = 
\left[\begin{array}{@{}*{4}{c}@{}}
     0&     1&     0&     0\\
     1&     0&     0&     0\\
     0&     0&     1&     0\\
     0&     0&     0&     1
  \end{array} \right]
\]

$\|Q^TQ-I\|_2$ = 4.224272170198187e-16

\newpage

\textbf{QR.m}
\verbatiminput{C:/Users/Christopher/Desktop/"MAT 128B"/"Project 3"/"QR.m"}

\newpage

\pdfbookmark[2]{(b)}{partb}
\textbf{(b)}

\[
b = 
\left[\begin{array}{@{}*{1}{c}@{}}
	1\\
	1\\
	1\\
	1\\
	1\\
	1
  \end{array} \right]
\]

\textbf{\underline{TOL = 10^{-16}}}

\[
x_0 =
\left[\begin{array}{@{}*{1}{c}@{}}
    -4.289142502257605e+14\\
    -8.578285004515210e+14\\
    -4.289142502257620e+14\\
     8.578285004515231e+14
  \end{array} \right]
\]

$\|b-Ax_0\|_2$ = 1.082531754730548e+00

\textbf{\underline{TOL = 10^{-14}}}

\[
x_0 =
\left[\begin{array}{@{}*{1}{c}@{}}
     9.999999999999997e-01\\
     1.999999999999999e+00\\
    -4.999999999999998e-01\\
                         0
  \end{array} \right]
\]

$\|b-Ax_0\|_2$ = 9.999999999999999e-01

\textbf{\underline{TOL = 10^{-12}}}

\[
x_0 =
\left[\begin{array}{@{}*{1}{c}@{}}
     9.999999999999997e-01\\
     1.999999999999999e+00\\
    -4.999999999999998e-01\\
                         0
  \end{array} \right]
\]

$\|b-Ax_0\|_2$ = 9.999999999999999e-01

For TOL = $10^{-16}$, we see that the entries of $x_0$ are very large which is alarming considering the entries of b are 1's and the entries of A are small integers. In fact, MATLAB prints an error that A is close to singular which we can see in cases TOL = $10^{-14}$ or TOL = $10^{-12}$ where the rank is 3 instead of 4 as in TOL = $10^{-16}$. To trust our numerical results, we would use TOL = $10^{-14}$ or TOL = $10^{-12}$.

\newpage

\textbf{QR\_solve.m}
\verbatiminput{C:/Users/Christopher/Desktop/"MAT 128B"/"Project 3"/"QR_solve.m"}

\newpage

\pdfbookmark[2]{(c)}{partc}
\textbf{(c)} Prove that the coefficient matrix $M = A^TA$ of the normal equations is symmetric positive definite if rank$A = n$.

\begin{proof}
Let $M = A^TA$. We first show that M is symmetric. It is clear that $M^T = (A^TA)^T = A^TA = M$. Thus, M is symmetric.

To show that $M$ is positive definite, we first recall the definition of a positive definite matrix. A matrix $M$ is positive definite iff $x^TMX > 0$ $\forall x\neq 0$. So we can write
\begin{align*}
	x^TMX &= x^TA^TAx\\
				&= (Ax)^T(Ax)\\
				&> 0.
\end{align*}
The last line comes from the fact that $Ax$ is a vector and its inner (dot) product with itself is always positive. $(Ax)^T(Ax)$ is also equal to the Euclidean norm of $Ax$ squared which is also positive (for $x \neq 0$).
\end{proof}

\newpage

\pdfbookmark[2]{(d)}{partd}
\textbf{(d)}

\textbf{Cholesky\_normal\_equations.m}
\verbatiminput{C:/Users/Christopher/Desktop/"MAT 128B"/"Project 3"/"Cholesky_normal_equations.m"}

\newpage

\pdfbookmark[2]{(e)}{parte}
\textbf{(e)} Using $A\in \mathbb{R}^{17\times 13}$ and $b\in \mathbb{R}^{17}$ given in the Matlab file prog3e.mat.

\textbf{\underline{TOL = 10^{-14}}}

\underline{QR}
\[
x =
\left[\begin{array}{@{}*{1}{c}@{}}
    -1.546070222421405e+11\\
    -9.297487747885947e+10\\
     1.493693105771897e+11\\
     1.066527222191558e+11\\
    -2.537212627334844e+11\\
     1.606536022062050e+11\\
     8.872212574108492e+10\\
    -7.871640990844116e+10\\
     1.954514525566362e+11\\
    -2.671291704890820e+10\\
     1.242255913092082e+11\\
    -2.309933234896546e+11\\
    -9.934598662102525e+10
  \end{array} \right]
\]

$\|b-Ax\|_2$ = 4.678432159367829e-05

\underline{$LDL^T$}
\[
x =
\left[\begin{array}{@{}*{1}{c}@{}}
		-4.112121011942238e-01\\
     5.472316353121219e-01\\
    -5.173550705393407e-01\\
     4.551617465426449e-01\\
    -2.703669723434581e-01\\
     5.772905380891625e-01\\
    -5.275351589993031e-01\\
    -8.210281428254646e-01\\
    -8.095436455955173e-01\\
     8.320110605315605e-02\\
    -4.355899005820045e-02\\
     7.370565021068745e-02\\
    -1.517278149777153e-01\\
    -6.337289071975470e-02\\
     1.486174812005002e-01\\
     2.815813274135337e-01\\
     1.787279076014642e-01
  \end{array} \right]
\]

$\|b-Ax\|_2$ = 6.685060541230082e-01

\newpage

\textbf{\underline{TOL = 10^{-12}}}

\underline{QR}
\[
x =
\left[\begin{array}{@{}*{1}{c}@{}}
    -1.546070222421405e+11\\
    -9.297487747885947e+10\\
     1.493693105771897e+11\\
     1.066527222191558e+11\\
    -2.537212627334844e+11\\
     1.606536022062050e+11\\
     8.872212574108492e+10\\
    -7.871640990844116e+10\\
     1.954514525566362e+11\\
    -2.671291704890820e+10\\
     1.242255913092082e+11\\
    -2.309933234896546e+11\\
    -9.934598662102525e+10
  \end{array} \right]
\]

$\|b-Ax\|_2$ = 4.678432159367829e-05

\underline{$LDL^T$}
\[
x =
\left[\begin{array}{@{}*{1}{c}@{}}
    -1.084212758253467e+07\\
    -5.341898337215079e+07\\
     5.402236389254467e+06\\
    -3.078563540776220e+07\\
    -7.317060218039244e+06\\
     7.979441421510610e+07\\
    -4.078790139396015e+07\\
    -7.445778967454656e+05\\
     4.631027351273581e+07\\
     6.041363102650469e+06\\
     3.048194590502472e+05\\
    -2.752119725406089e+07\\
    -1.462576296252692e+07
  \end{array} \right]
\]

$\|b-Ax\|_2$ = 6.685060541230082e-01

\newpage

\textbf{\underline{TOL = 10^{-10}}}

\underline{QR}
\[
x =
\left[\begin{array}{@{}*{1}{c}@{}}
    -1.279333791949196e+08\\
     7.989637293852357e+08\\
     1.788154841354073e+09\\
     9.606338899267700e+08\\
     1.092935711880787e+08\\
    -2.566553685570429e+08\\
    -3.070062117655145e+09\\
     2.373770402763600e+09\\
    -7.070470804156628e+08\\
    -1.541995609142110e+09\\
    -2.129347940478388e+09\\
                         0\\
                         0
  \end{array} \right]
\]

$\|b-Ax\|_2$ = 5.565509038236765e-01

\underline{$LDL^T$}
Does not exist for this tolerance because at this level, rank(A) = 11 $<$ 13. That is, A is not full rank at TOL = 10^{-10}.

\newpage

\pdfbookmark[2]{(f)}{partf}
\textbf{(f)}

\textbf{\underline{Theory 1}}

\[
	f_1(t) = c_0 + c_1te^{-t} + c_2t^2e^{-2t}
\]

\[
x =
\left[\begin{array}{@{}*{1}{c}@{}}
	  c_0\\
		c_1\\
		c_2
  \end{array} \right]
=
\left[\begin{array}{@{}*{1}{c}@{}}
	   9.989466497393511e-01\\
    -9.982112073329603e-01\\
    -9.965726661416167e-01
  \end{array} \right]
\]

\textbf{\underline{Theory 2}}

\[
	f_2(t) = c_0 + c_1\sqrt{t}e^{-\sqrt{t}} + c_2te^{-2\sqrt{t}}
\]

\[
x =
\left[\begin{array}{@{}*{1}{c}@{}}
	  c_0\\
		c_1\\
		c_2
  \end{array} \right]
=
\left[\begin{array}{@{}*{1}{c}@{}}
     5.845428014785350e-01\\
     4.247465387007817e+00\\
    -1.204915107205322e+01
  \end{array} \right]
\]

\begin{figure}[hbtp]
  \begin{center}
    \includegraphics[width=.8\textwidth, height = .4\textheight]{"Partf".png}
    \caption{}
  \end{center}
\end{figure}

Theory 1 is more appropriate for the given data because its least squares line follows the data better than Theory 2.

\end{document}