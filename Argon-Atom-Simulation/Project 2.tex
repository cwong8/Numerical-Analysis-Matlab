\documentclass[final,12pt,reqno]{amsart}

\setlength{\textwidth }{7.50 in}
\setlength{\textheight}{9.25 in}
\setlength{\oddsidemargin }{0.00 in}
\setlength{\evensidemargin}{0.00 in}
\setlength{\oddsidemargin }{0.00 in}
\setlength{\evensidemargin}{0.00 in}
\setlength{\hoffset}{-0.50 in}
\setlength{\voffset}{-0.50 in}
\setlength{\headsep}{12 pt}
\setlength{\headheight}{40.53336 pt}
\setlength{\topmargin }{00 pt}
\setlength{\footskip}{0.50 in}
\setlength{\parskip}{12 pt}
\setlength{\parindent}{00 pt}
\setlength{\fboxsep}{10 pt}

\usepackage{graphicx}
\DeclareGraphicsExtensions{.png}
\graphicspath{{C:/Users/Christopher Wong/Desktop/''MAT 128C''/''Project 2''}}

\usepackage{enumerate}
\usepackage{multicol}
\usepackage{bm}
\usepackage{color}

\usepackage[bookmarks=true]{hyperref}
\usepackage{bookmark}

\usepackage{verbatim}
\usepackage{slashbox}
\usepackage{mathtools}
\usepackage{units}

\font\myfiverm=cmr5 scaled 500

\usepackage{fancyhdr}
\pagestyle{fancy}
\fancyhead{}
\fancyhead[LO]{}
\fancyhead[LE]{}
\fancyhead[CO]{\textbf{Project 2}}
\fancyhead[CE]{\textbf{Project 2}}
\fancyhead[RO]{\textbf{Christopher Wong\\
                        999234204\\
												MAT 128C}}
\fancyhead[RE]{\textbf{Christopher Wong\\
                        999234204\\
												MAT 128C}}
												
\newcommand\abs[1]{\left|#1\right|}
\renewcommand{\arraystretch}{1.2}

%%%%%%%%%%%%%%%%%%%%%%% START OF DOCUMENT %%%%%%%%%%%%%%%%%%%%%%%

\begin{document}

\thispagestyle{fancy}

\pdfbookmark[1]{Project 2.m}{project2}

\textbf{Project\_2.m} takes in
\newline
	N = \text{Number of atoms in the system}
	\newline
	L = \text{Size of the system (boundaries of} $\left[0, 2\sigma L\right]$)
	\newline
	T = \text{Number of time steps}

with $\Delta T$ = 2 picoseconds and $\sigma$ = 1. It then calculates and stores the positions of each Argon atom at each time step in a matrix $x$ and $y$ for the x and y coordinates of the atom, respectively. Then it plots the results. We set the seed with $rng(123)$ and simulate 500 atoms in a $(\left[0, 2\right] \times \left[0, 2\right])$ system. The first 10 picoseconds are given below:
\begin{center}
	\includegraphics[height=4in]{{t0}.png}
\end{center}

\begin{center}
	\includegraphics[height=4in]{{t1}.png}
\end{center}
\begin{center}
	\includegraphics[height=4in]{{t2}.png}
\end{center}
\begin{center}
	\includegraphics[height=4in]{{t3}.png}
\end{center}
\begin{center}
	\includegraphics[height=4in]{{t4}.png}
\end{center}
\begin{center}
	\includegraphics[height=4in]{{t5}.png}
\end{center}

I am not sure if my calculated temperatures are correct because the velocities go from $v_0 = 0.1$ to $v_1 > 10^9$ as a result of dividing by $2\Delta T$ = ($4*10^{-12}$ seconds) and thus the temperatures end up as:

\begin{center}
	\begin{tabular}{|c|c|}
		\hline
		Time (ps) & Temperature (K)\\
		\hline
		0 & 7.655274297457165e-03\\
		\hline
		2 & 7.410371621844270e+50\\
		\hline
		4 & 6.837365137519428e+42\\
		\hline
		6 & 1.306611796774299e+47\\
		\hline
		8 & 5.967677920824267e+57\\
		\hline
		10 & 3.250203995660685e+50\\
		\hline
	\end{tabular}
\end{center}

\newpage

\textbf{Project\_2.m}

\verbatiminput{C:/Users/"Christopher Wong"/Desktop/"MAT 128C"/"Project 2"/"Project_2.m"}


\end{document}



